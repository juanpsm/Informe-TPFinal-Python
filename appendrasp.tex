\section{Raspeberry Pi}\label{raspeberry-pi}

\subsection{Primer aproximación}\label{primer-aproximacion}

\begin{itemize}
\itemsep1pt\parskip0pt\parsep0pt
\item
  Simulador web de Sensor HAT (ya tiene led y sensores integrados):\\
  \url{https://trinket.io/library/trinkets/5b83aa39e6}\\ Es mas simple
  pero para probar algunos comandos está piola. La desventaja es que no
  se puede instalar cualquier librería. Sirve como una primera
  aproximación a una Raspberry Pi.
\end{itemize}

\subsection{Probando el sistema
\textbf{Raspbian}}\label{probando-el-sistema-raspbian}

Para simular la \textbf{Raspberry} se puede usar
\href{https://www.virtualbox.org/wiki/Downloads}{VirtualBox} e instalar
una maquina con \href{https://www.raspberrypi.org/downloads}{Raspbian},
o en Windows directamente usar
\href{https://www.qemu.org/download/}{Qemu}. Para mi anda mejor
VirtualBox porque le podes asignar más memoria ram al emulador. Mas
informacion de como instalar la máquina virtual
\href{https://thepi.io/how-to-run-raspberry-pi-desktop-on-windows-or-macos/}{aqui}

Para tener copy paste entre el sistema anfitrión y la maquina virtual,
insertar el cd de adicionales desde el menu \emph{Devices} y luego en
una terminal:\\\texttt{\$ sh /media/cdrom/VBoxLinuxAdditions.run}

\subsection{Instalación de requisitos}\label{instalacion-de-requisitos}

\begin{itemize}
\itemsep1pt\parskip0pt\parsep0pt
\item
  \href{https://sourceforge.net/p/raspberry-gpio-python/wiki/install/}{RPi.GPIO Installation}
  
\begin{cverbatim}
$ sudo apt-get update
$ sudo apt-get install python-rpi.gpio python3-rpi.gpio
\end{cverbatim}
\item
  \href{https://github.com/adafruit/Adafruit_Python_DHT\#installing}{Adafruit\_Python\_DHT Installation}
  \begin{cverbatim}
$ sudo apt-get install python-pip
$ sudo python -m pip install --upgrade pip setuptools wheel
$ sudo pip install Adafruit_DHT
  \end{cverbatim}
  
\item
  \href{https://luma-led-matrix.readthedocs.io/en/latest/install.html}{Luma.LED\_Matrix: Display drivers for MAX7219, WS2812, APA102}
  \begin{cverbatim}
$ sudo usermod -a -G spi,gpio pi
$ sudo apt-get install build-essential python-dev python-pip libfreetype6-dev libjpeg-dev
$ sudo -H pip install --upgrade --ignore-installed pip setuptools
$ sudo -H pip install --upgrade luma.led_matrix
  \end{cverbatim}

\item
  \href{https://github.com/rm-hull/luma.examples\#installation-instructions}{Luma.Examples}
  \begin{cverbatim}
$ sudo usermod -a -G i2c,spi,gpio pi
$ sudo apt install python-dev python-pip libfreetype6-dev libjpeg-dev build-essential
$ sudo apt install libsdl-dev libportmidi-dev libsdl-ttf2.0-dev libsdl-mixer1.2-dev libsdl-image1.2-dev
  \end{cverbatim}

\item Log out y volver a entrar. clonar repo:
\begin{cverbatim}
$ git clone https://github.com/rm-hull/luma.examples.git
$ cd luma.examples
  \end{cverbatim}

\item Instalar las librerias:
  \begin{cverbatim}
$ sudo -H pip install -e .
  \end{cverbatim}

\item{Correr ejemplos:}
\begin{cverbatim}
$ python luma.examples/examples/3d_box.py
  \end{cverbatim}
 
(esto último solo andará en
una verdadera Raspberry Pi, ya que hace uso de los sensores y
dispositivos que el emulador no posee \emph{a priori}. Para ello
dirigirse a la sección \ref{emu})


\end{itemize}
\subsection{Usando los sensores y
dispositivos}\label{usando-los-sensores-y-dispositivos}

\subsubsection{Trabajar con \textbf{GPIO} (General Purpose I/O, los
pines de la
placa):}\label{trabajar-con-gpio-general-purpose-io-los-pines-de-la-placa}

\begin{itemize}
\item
  Pinout:
  \url{https://github.com/splitbrain/rpibplusleaf}\\\includegraphics{https://user-images.githubusercontent.com/11953173/59568951-fa28a400-9058-11e9-8a33-0915e46e13a7.png}\\
\item
  Interactivo: \url{https://pinout.xyz/pinout/pin33_gpio13}\\
\item
  Uso Basico (esto lo vamos a usar para el \textbf{microfono}):\\
  \url{https://sourceforge.net/p/raspberry-gpio-python/wiki/BasicUsage/}\\
  \url{https://www.raspberrypi.org/documentation/usage/gpio/}\\
  ```pyhton\\ import RPi.GPIO as GPIO

  GPIO.setmode(GPIO.BOARD)\\ \# or\\ GPIO.setmode(GPIO.BCM)

  GPIO.setup(channel, GPIO.IN)

  \# Read\\ GPIO.input(channel)

  \# Set\\ GPIO.output(channel, state)

  \# Poll\\ if GPIO.input(channel):\\ print(`Input was HIGH')\\ else:\\
  print(`Input was LOW')

  \# To wait for a button press by polling in a loop:\\ while
  GPIO.input(channel) == GPIO.LOW:\\ time.sleep(0.01) \# wait 10 ms to
  give CPU chance to do other things

  \# To clean up at the end of your script:\\ GPIO.cleanup()\\ ```
\end{itemize}

\subsubsection{Para el \textbf{sensor} de temp y
humedad:}\label{para-el-sensor-de-temp-y-humedad}

\includegraphics{https://user-images.githubusercontent.com/11953173/59585062-02fb9300-90b6-11e9-9107-9b8ddbdd11bb.jpg}

\url{https://github.com/adafruit/Adafruit_Python_DHT/blob/master/examples/simpletest.py}\\
```python\\ import Adafruit\_DHT\\ sensor = Adafruit\_DHT.DHT22\\ pin =
23

hum, temp = Adafruit\_DHT.read\_retry(sensor, pin)\\ ```

\subsubsection{Para mostrar la info en las \textbf{matrices de
LED}:}\label{para-mostrar-la-info-en-las-matrices-de-led}

\url{https://luma-led-matrix.readthedocs.io/en/latest/python-usage.html\#x8-led-matrices}\\\url{https://github.com/rm-hull/luma.examples}\\
```python\\ from luma.core.interface.serial import spi, noop\\ from
luma.led\_matrix.device import max7219\\ from luma.core.render import
canvas\\ from luma.core.legacy import text, show\_message\\ from
luma.core.legacy.font import proportional, CP437\_FONT, TINY\_FONT,
SINCLAIR\_FONT, LCD\_FONT\\ from luma.core.virtual import viewport

serial = spi(port=0, device=0, gpio=noop())\\ device = max7219(serial,
cascaded=2, block\_orientation=-90)\\ device.contrast(0x05)\\ msg =
`asd'\\ \# para mostrar un mensage que vaya pasando usar
show\_message:\\ show\_message(device, msg, fill=``white'',
font=proportional(LCD\_FONT), scroll\_delay=0.05)

\# para mostrar un mensaje estático usar draw y time para ir
actualizándolo:\\ while True: \# o import repeat y repeat(None).\\
time.sleep(1)\\ msg = time.asctime()\\ msg= time.strftime(``\%H\%M'')\\
with canvas(device) as draw:\\ text(draw, (1, 0), msg, fill=``white'')\\
time.sleep(2)\\ pass \# ???\\ ```

\begin{itemize}
\itemsep1pt\parskip0pt\parsep0pt
\item
  para dibujar en las celdas y obtener el codigo:\\
  \url{http://dotmatrixtool.com/}
\end{itemize}

\subsubsection{Para \emph{emular}:}\label{para-emular}

Primero instalamos
\href{https://luma-emulator.readthedocs.io/en/latest/index.html}{Luma.emulation}

\begin{verbatim}
$ sudo apt install python-dev python-pip build-essential
$ sudo apt install libsdl-dev libportmidi-dev libsdl-ttf2.0-dev libsdl-mixer1.2-dev libsdl-image1.2-dev
$ sudo pip install --upgrade luma.emulator
\end{verbatim}

(recordar hacer lo mismo con \textbf{python3} y \textbf{pip3})

Hay que correr los archivos con unos parámetros especiales. Los más
importantes son:

\begin{verbatim}
   --display DISPLAY, -d DISPLAY
                        Display type, supports real devices or emulators.
                        Allowed values are: ssd1306, ssd1309, ssd1322,
                        ssd1325, ssd1327, ssd1331, ssd1351, sh1106, pcd8544,
                        st7735, ht1621, uc1701x, st7567, max7219, ws2812,
                        neopixel, neosegment, apa102, capture, gifanim,
                        pygame, asciiart, asciiblock (default: ssd1306)
\end{verbatim}

De estas opciones notar:

\begin{itemize}
\itemsep1pt\parskip0pt\parsep0pt
\item
  max7219: Este es cuando tengamos el dispositivo real.\\
\item
  capture: Este saca una instanteanea de lo que mostraria el display y
  lo guarda como png.\\
\item
  gifanim: Este es como capture pero guarda un gif animado.\\
\item
  \textbf{pygame}: Este muestra el output en una ventana en tiempo real
\end{itemize}

Seguimos con los parametros restantes

\begin{verbatim}
   --width WIDTH         Width of the device in pixels (default: 128)
   
   --height HEIGHT       Height of the device in pixels (default: 64)
   
   --rotate ROTATION, -r ROTATION
                     Rotation factor. Allowed values are: 0, 1, 2, 3
                     (default: 0)
   --transform TRANSFORM
                     Scaling transform to apply (emulator only). Allowed
                     values are: identity, led_matrix, none, scale2x,
                     seven_segment, smoothscale (default: scale2x)
    --scale SCALE         Scaling factor to apply (emulator only) (default: 2)
\end{verbatim}

\subsection{Por ejemplo, para simular dos modulos de 8x8 en matriz de
led en tiempo
real:}\label{por-ejemplo-para-simular-dos-modulos-de-8x8-en-matriz-de-led-en-tiempo-real}

El codigo hay que cambiarlo un poco:

\begin{Shaded}
\begin{Highlighting}[]
\CommentTok{# from luma.core.interface.serial import spi, noop}
\CommentTok{# from luma.led_matrix.device import max7219}
\CharTok{from} \NormalTok{luma.core.legacy }\CharTok{import} \NormalTok{text, show_message}
\CharTok{from} \NormalTok{luma.core.legacy.font }\CharTok{import} \NormalTok{proportional, CP437_FONT, TINY_FONT, SINCLAIR_FONT, LCD_FONT}
\CharTok{from} \NormalTok{demo_opts }\CharTok{import} \NormalTok{get_device}

\CommentTok{# comentar las lineas:}
\CommentTok{# serial = spi(port=0, device=0, gpio=noop())}
\CommentTok{# device = max7219(serial, cascaded=2, block_orientation=-90)}

\NormalTok{device = get_device()}

\NormalTok{msg = }\StringTok{'Aguante Python y la UNLP 2k19'}
\NormalTok{show_message(device, msg, fill=}\StringTok{"white"}\NormalTok{, font=proportional(LCD_FONT), scroll_delay=}\FloatTok{0.05}\NormalTok{)}
\end{Highlighting}
\end{Shaded}

Y ejecutarlo con los parámetros:

\texttt{\$ python luma.examples/examples/archivito.py -{}-display pygame -{}-transform led\_matrix -{}-width 16 -{}-height 8}

\includegraphics{https://user-images.githubusercontent.com/11953173/59584554-b2376a80-90b4-11e9-85f7-b97ebf2d9019.gif}

\textbar{} \emph{Nota}: ya que hay que usar las librerías demo lo mas
facil es meter el archivo en la carpeta \emph{luma.examples/examples}
que se creó cuando clonamos el git \textbar{}\\\textbar{} --- \textbar{}
